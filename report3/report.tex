\documentclass[uplatex]{jsarticle}
\和暦
\usepackage[top=25truemm,bottom=25truemm,left=25truemm,right=25truemm]{geometry}
\usepackage[dvipdfmx]{graphicx}
\usepackage{color}
\usepackage{authblk}
\usepackage{amsmath, amssymb}

\title{\huge 数理統計学特論}
\author{199X223X 綿岡晃輝}
\affil{神戸大学大学院 システム情報学研究科 計算科学専攻}
\date{}

\begin{document}
    \maketitle
    \newpage
    
    \section{BIBD(11, 11, 5, 5, 2)とBIBD(13, 13, 4, 4, 1)を作れ.}
    \subsection{BIBD(11, 11, 5, 5, 2)}
    $V := \mathbb{Z}/11\mathbb{Z}=\{0, 1, 2, 3, 4, 5, 6, 7, 8, 9, 10\}$とする. \\
    V = \{ \\
    \{2,    4,    7,   10,   0 \}, \\ 
    \{5,    6,    7,    9,   0 \}, \\ 
    \{4,    5,    8,    9,  10 \}, \\
    \{1,    6,    8,   10,   0 \}, \\ 
    \{3,    4,    6,    7,   8 \}, \\
    \{2,    3,    6,    9,  10 \}, \\
    \{1,    3,    4,    9,   0 \}, \\ 
    \{2,    3,    5,    8,   0 \}, \\ 
    \{1,    3,    5,    7,  10 \}, \\
    \{1,    2,    7,    8,   9 \}, \\
    \{1,    2,    4,    5,   6 \}, \\
    \}

    \subsection{BIBD(13, 13, 4, 4, 1)}
    $V := \mathbb{Z}/13\mathbb{Z}=\{0, 1, 2, 3, 4, 5, 6, 7, 8, 9, 10, 11, 12\}$とする. \\
    V = \{ \\
    \{1,    2,    5,   7 \}, \\
    \{3,    5,   10,  11 \}, \\
    \{5,    6,   12,   0 \}, \\
    \{7,    9,   10,  12 \}, \\
    \{7,    8,   11,   0 \}, \\
    \{1,    6,    8,  10 \}, \\
    \{4,    5,    8,   9 \}, \\
    \{3,    4,    6,   7 \}, \\
    \{1,    4,   11,   2 \}, \\
    \{1,    3,    9,   0 \}, \\
    \{2,    4,   10,   0 \}, \\
    \{2,    6,    9,  11 \}, \\
    \{2,    3,    8,  12 \}, \\
    \}

    \newpage

    \section{$z^2=6x^2-y^2$は自明な解しか持たないことを示せ.}
    \noindent
    $z^2=6x^2-y^2$は自明な解以外を持つと仮定し, 矛盾を導く. \\
    任意の整数kに対して, \\
    \begin{eqnarray}
        & (3k+1)^2 = 9k^2 + 6k + 1  \equiv 1\ (mod.3) \\
        & (3k+2)^2 = 9k^2 + 12k + 4  \equiv 1\ (mod.3)
    \end{eqnarray}
    が言える. \\
    \begin{align*}
        x=3^\alpha x' \ (\alpha \in {0} \cup \mathbb{N}, x':3の倍数でない整数) \\
        y=3^\beta y' \ (\beta \in {0} \cup \mathbb{N}, y':3の倍数でない整数) \\
        z=3^\gamma z' \ (\gamma \in {0} \cup \mathbb{N}, z':3の倍数でない整数) \\
    \end{align*}
    とおき, 一般性を失わないので, $\beta \geq \gamma$とすると,
    \begin{align*}
        6x^2 &= 6(3^\alpha x')^2 \\
        &= 3^{2\alpha + 1}\{2(x')^2\} \\
    \end{align*}
    となり, 
    \begin{align*}
        y^2 + z^2 &= (3^{\beta}y')^2 + (3^{\gamma}z')^2 \\
        &= 3^{2\gamma} \{(3^{\beta - \gamma}y')^2 + (z')^2\}
    \end{align*}
    となる. \\
    ここで, $x'$は3の倍数ではないため, (1),(2)より$2(x'^2) \equiv 2\ (mod.3)$となる. \\
    また, $(3^{\beta - \gamma}y')^2 \equiv 0, (z')^2 \equiv 1\ (mod.3)$なので,
    $(3^{\beta - \gamma}y')^2 + (z')^2 \equiv 1\ (mod.3)$となる.\\
    つまり, $2(x')^2$も$(3^{\beta - \gamma}y')^2 + (z')^2$も3の倍数にはならない. \\
    故に, $6x^2$と$y^2+z^2$を素因数分解した時, 3の累乗が同じにならないため, 
    \begin{align*}
        6x^2 \neq y^2 + z^2 
    \end{align*}
    となり, 矛盾. \\
    \noindent
    従って,$z^2=6x^2-y^2$は自明な解しか持たない.

\end{document}